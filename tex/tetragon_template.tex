\documentclass[10pt,a4paper]{article}       % А5 бумага, шрифт 10

\usepackage[top=2cm, left=1cm, right=1cm, bottom=2cm]{geometry} % отступыы
\usepackage{array, amsmath, amstext, multicol, amsfonts, nccmath, mathabx, dsfont, amssymb, amsfonts}     % математические пакеты
\usepackage{cmap}					    % поиск в PDF
\usepackage[T2A]{fontenc}			    % кодировка
\usepackage[utf8]{inputenc}			    % кодировка исходного текста
\usepackage[english,russian]{babel}	    % локализация и переносы
\usepackage[unicode, pdftex]{hyperref}  % настройки ссылок и PDF
\usepackage{graphicx}                   % картинки
\graphicspath{{pictures/}}              % картинки
\DeclareGraphicsExtensions{.pdf,.png,.jpg} % картинки
\usepackage[unicode]{hyperref}
\usepackage{subcaption}
\usepackage{adjustbox}
\usepackage{pdflscape}
\usepackage{longtable}
\usepackage{pdflscape}
\usepackage{graphicx}
\usepackage{everypage}
\usepackage{pdflscape}
\usepackage{tikz}
\usepackage[outputdir=build]{minted}
\usepackage{titlesec}
\usepackage{fancyhdr}
\usepackage{enumitem}
\usepackage{lscape}
\usetikzlibrary{shapes,arrows}
\title{Тетрагон}
\author{Tetragon Drawer}
\date{\today}

\def\skippar{2cm}
\def\tkizx{2.5}
\def\tkizy{2.5}
\definecolor{bg}{rgb}{0.95,0.95,0.92}
\definecolor{darkgreen}{rgb}{0,0.6,0}
\definecolor{red2green}{rgb}{1,0.6,0}
\definecolor{blue2magenta}{rgb}{0.5,0,1}
\DeclareCaptionLabelFormat{custom}{}
\setlength{\parindent}{0.5cm}
\newcounter{taskid}
\newcounter{point}
\fancypagestyle{plain}{\fancyhf{}\renewcommand{\headrulewidth}{0pt}}
\pagestyle{fancy}

\def\mymark{}
\fancyhf{}
\rfoot{\thepage}
\chead{\mymark}
\def\oneout#1{\textit{#1}}
\def\sepout{ $\bullet$}
\newcommand\upfooter[1]{\renewcommand\mymark{\foreach{\oneout}{\sepout}{#1}}}

\def\header{
    \titleformat{\section}[display]{\filcenter}{}{12pt}{\bfseries}{}
    \titleformat{\subsection}[display]{\filcenter}{}{12pt}{\bfseries}{}
    \titleformat{\subsubsection}[display]{\filcenter}{}{12pt}{\bfseries}{}
    \captionsetup{labelformat=custom}
    \renewcommand{\contentsname}{\Large Содержание}
}

\def\task{\textbf{№ \arabic{taskid}. }\stepcounter{taskid}}
\def\point0{

\setlength{\parindent}{0cm}\setcounter{point}{1}\leftskip=1cm}
\def\pointi{

\textbf{\alph{point}) }\stepcounter{point}}
\def\pointn{

\setlength{\parindent}{0.5cm}\leftskip=0cm}
\def\step0{\setcounter{point}{1}}
\def\stepi{\textbf{(\alph{point}) }\stepcounter{point}}

\def\headS#1{
    \section{\Large #1}
    \indent
    
}

\def\head#1{
	\vspace*{-2cm}
    \setcounter{taskid}{1}
    \subsection{\Large #1}
    \indent
    
}

\def\subhead#1{
	\subsubsection{\Large #1}
	\indent
	
}





\makeatletter

\def\foreach#1#2#3{%
  \@test@foreachA{#1}{#2}#3,\@end@token%
}

\def\@swallow#1{}

\def\@test@foreachA#1#2{%
  \@ifnextchar\@end@token%
    {\@swallow}%
    {\@foreach{#1}{#2}}%
}

\def\@test@foreachB#1#2{%
  \@ifnextchar\@end@token%
    {\@swallow}%
    {#2 \@foreach{#1}{#2}}%
}

\def\@foreach#1#2#3,#4\@end@token{%
  #1{#3}%
  \@test@foreachB{#1}{#2}#4\@end@token%
}

\makeatother	


\begin{document}
\begin{landscape}
\header



\begin{figure}[H]
\centering
\begin{tikzpicture}[>=latex, xscale=\tkizx, yscale=\tkizy]
	\node[rectangle, draw, darkgreen, inner sep=5pt] (a) at (5,7) {\color{red} stra};
	\node[rectangle, draw, darkgreen, inner sep=5pt] (b) at (11,7) {\color{red} strb};
	\node[rectangle, draw, darkgreen, inner sep=5pt] (c) at (11,1) {\color{red} strc};
	\node[rectangle, draw, darkgreen, inner sep=5pt] (d) at (5,1) {\color{red} strd};
	\path (a) to [bend left=15] node (ab) {~} (b);
	\path (a) to [bend left=15] node (ac) {~} (c);
	\path (a) to [bend left=15] node (ad) {~} (d);
	\path (b) to [bend left=15] node (ba) {~} (a);
	\path (b) to [bend left=15] node (bc) {~} (c);
	\path (b) to [bend left=15] node (bd) {~} (d);
	\path (c) to [bend left=15] node (ca) {~} (a);
	\path (c) to [bend left=15] node (cb) {~} (b);
	\path (c) to [bend left=15] node (cd) {~} (d);
	\path (d) to [bend left=15] node (da) {~} (a);
	\path (d) to [bend left=15] node (db) {~} (b);
	\path (d) to [bend left=15] node (dc) {~} (c);
	\draw[blue, ->] (a) -- (ab);
	\draw[blue, ->] (ab) -- (b);
	\draw[blue, ->] (a) -- (ac);
	\draw[blue, ->] (ac) -- (c);
	\draw[blue, ->] (a) -- (ad);
	\draw[blue, ->] (ad) -- (d);
	\draw[blue, ->] (b) -- (ba);
	\draw[blue, ->] (ba) -- (a);
	\draw[blue, ->] (b) -- (bc);
	\draw[blue, ->] (bc) -- (c);
	\draw[blue, ->] (b) -- (bd);
	\draw[blue, ->] (bd) -- (d);
	\draw[blue, ->] (c) -- (ca);
	\draw[blue, ->] (ca) -- (a);
	\draw[blue, ->] (c) -- (cb);
	\draw[blue, ->] (cb) -- (b);
	\draw[blue, ->] (c) -- (cd);
	\draw[blue, ->] (cd) -- (d);
	\draw[blue, ->] (d) -- (da);
	\draw[blue, ->] (da) -- (a);
	\draw[blue, ->] (d) -- (db);
	\draw[blue, ->] (db) -- (b);
	\draw[blue, ->] (d) -- (dc);
	\draw[blue, ->] (dc) -- (c);
\end{tikzpicture}
% \caption{\small Подпись к картинке}
\label{fig:graph}
\end{figure}


\end{landscape}
\end{document}